% Created 2022-05-17 Tue 15:06
% Intended LaTeX compiler: pdflatex
\documentclass[11pt]{article}
\usepackage[utf8]{inputenc}
\usepackage[T1]{fontenc}
\usepackage{graphicx}
\usepackage{longtable}
\usepackage{wrapfig}
\usepackage{rotating}
\usepackage[normalem]{ulem}
\usepackage{amsmath}
\usepackage{amssymb}
\usepackage{capt-of}
\usepackage{hyperref}
\usepackage{xcolor}
\usepackage{times}\hypersetup{
colorlinks,
linkcolor={red!80!black},
citecolor={blue!50!black},
urlcolor={blue!80!black}
}
\author{Spencer Smith}
\date{\today}
\title{Projectile Motion Lesson}
\hypersetup{
 pdfauthor={Spencer Smith},
 pdftitle={Projectile Motion Lesson},
 pdfkeywords={},
 pdfsubject={},
 pdfcreator={Emacs 28.1 (Org mode 9.5.2)}, 
 pdflang={English}}
\begin{document}

\maketitle
\tableofcontents

We structure this lesson following Section 12.6 (Motion of a
Projectile) from the classic Hibbler text "Engineering Mechanics
Dynamnics, 10th edition".

\section{Learning Objectives}
\label{sec:org335c56f}

\begin{itemize}
\item Derive kinematic equations for 2D projectile motion from kinematic
equations from 1D rectilinear motion
\item Identify the assumptions required for the projectile motion
equations to hold:
\item Air resistance is neglected
\begin{itemize}
\item Gravitational acceleration acts downward and is constant,
regardless of altitude
\end{itemize}
\item Solve any given (well-defined) free-flight projectile motion
problems by:
\begin{itemize}
\item Able to select an appropriate Cartesian coordinate system to
simplify the problem as much as possible
\item Able to identify the known variables
\item Able to identify the unknown variables
\item Able to write projectile motion equations for the given problem
\item Able to solve the projectile motion equations for the unknown
quantities
\end{itemize}
\end{itemize}


\section{Rectilinear Kinematics: Continuous Motion (Recap)}
\label{sec:org10768b5}

As covered previously, the equations relating velocity (\(v\)), position
(\(p\)) and time (\(t\)) for motion in one dimension with constant
acceleration (\(a\)) are as follows:

\begin{equation}
\label{Eq_rectVel}
v = v^i + a t
\end{equation}

\begin{equation}
\label{Eq_rectPos}
p = p^i + v^i t + \frac{1}{2} a t^2
\end{equation}

\begin{equation}
\label{Eq_rectNoTime}
v^2 = (v^i)^2 + 2 a (p - p^i)
\end{equation}

where \(v^i\) and \(p^i\) are the initial velocity and position,
respectively.

Only two of these equations are independent, since the third equation
can always be derived from the other two.

\section{Motion of a Projectile}
\label{sec:orga8f3cda}

The free flight motion of a projectile is often studied in terms of
its rectangular components in the \(x-y\) plane, since the projectile's
acceleration \textbf{always} acts in the vertical direction.  To illustrate
the kinematic analysis, consider a projectile launched at point
\((p_x^i, p_y^i)\) with the initial velocity \(\mathbf{v}^i\) having
components \(v_x^i\) and \(v_y^i\), as shown in Figure \ref{fig:org28d7ddb}.
The position vector \(\mathbf{p}\) changes over time.  At any instant of
time \(t\) the position is represented by the components \(p_x\) and
\(p_y\). When we assume that air resistance is neglected, the only force
acting on the projectile is its weight, which causes the projectile to
have a \textbf{constant downward acceleration} of approximately \(a = a_y = -g
= -9.81 \text{m}/\text{s}^2\) or \(g = 32.2 \text{ft}/\text{s}^2\).\footnote{: This assumes that the earth's gravitational field does not vary
with altitude}


\begin{figure}[htbp]
\centering
\includegraphics[width=0.7\textwidth]{../CoordSystAndAssumpts.png}
\caption{\label{fig:org28d7ddb}Coordinate System and Definition of Symbols}
\end{figure}

The equations for rectilinear kinematics given above (Equations
\ref{Eq_rectVel}, \ref{Eq_rectPos} and \ref{Eq_rectNoTime}) are in
one dimension.  These equations can be applied for both the vertical
motion and the horizontal directions, as follows:

\subsection{Horizontal Motion}
\label{sec:orgee92c22}

For projectile motion the acceleration in the horizontal direction is
constant and equal to zero (\(a_x = 0\)).  This value can be substituted
in the equations for constant acceleration given above to yield the
following:

From Equation \ref{Eq_rectVel}:
\begin{equation}
\label{Eq_horizVel}
v_x = v_x^i
\end{equation}

From Equation \ref{Eq_rectPos}:
\begin{equation}
\label{Eq_horizPos}
p_x = p_x^i + v_x^i
\end{equation}

From Equation \ref{Eq_rectNoTime}:
\begin{equation}
\label{Eq_horizNoTime}
v_x = v_x^i
\end{equation}

Since the acceleration in the \(x\) direction (\(a_x\)) is zero, the
horizontal component of velocity always remains constant during
motion.

\subsection{Vertical Motion}
\label{sec:org94d95dc}

Since the positive \(y\) axis is directed upward, the acceleration in
the vertical direction is \(a_y = -g\).  This value can be substituted
in the equations for constant acceleration given above to yield the
following:

From Equation \ref{Eq_rectVel}:
\begin{equation}
v_y = v_y^i - g t
\label{Eq_vertVel}
\end{equation}

From Equation \ref{Eq_rectPos}
\begin{equation}
p_y = p_y^i + v_y^i t - \frac{1}{2} g t^2
\label{Eq_vertPos}
\end{equation}

From Equation \ref{Eq_rectNoTime}:
\begin{equation}
v_y^2 = (v_y^i)^2 - 2 g ( p_y - p_y^i)
\label{Eq_vertNoTime}
\end{equation}

Recall that the last equation can be formulated on the basis of
eliminating the time \(t\) between the first two equations and
therefore \textbf{only two of the above three equations are independent of
one another}.

\subsection{Summary}
\label{sec:org7b93a7e}

In addition to knowing that the horizontal component of velocity is
constant, problems involving the motion of a projectile can have at
most three unknowns since only three independent equations can be
written: that is, \textbf{one} equation in the \textbf{horizontal direction} and
\textbf{two} in the \textbf{vertical direction}.  Once \(v_x\) and \(v_y\) are obtained,
the resultant velocity \(\mathbf{v}\), which is \textbf{always tangent} to the
path, is defined by the \textbf{vector sum} as shown in Figure \ref{fig:org28d7ddb}.

\subsection{Procedure for Analysis}
\label{sec:orgcd65349}

Free-flight projectile motion problems can be solved using the
following procedure.

\subsubsection{Step 1: Coordinate System}
\label{sec:org98dcf1d}

\begin{itemize}
\item Establish the fixed \(x\), \(y\) coordinate axes and sketch the
trajectory of the particle.  Between any \textbf{two points} on the path
specify the given problem data and the \textbf{three unknowns}.  In all
cases the acceleration of gravity acts downward.  The particle's
initial and final velocities should be represented in terms of their
\(x\) and \(y\) components.

\item Remember that positive and negative position, velocity, and
acceleration components always act in accordance with their
associated coordinate directions.

\item The two points that are selected should be significant points where
something about the motion of the particle is known.  Potential
significant points include the initial point of launching the
projectile and the final point where it lands. The landing point
often has a known \(y\) value.

\item The variables in the equation may need to be changed to match the
notation of the specific problem.  For instance, a distinction may
need to be made between the \$x\$-coordinate of points \(A\) and \(B\),
via notation like \(p_x^A\) and \(p_x^B\).
\end{itemize}

\subsubsection{Step 2: Identify Knowns}
\label{sec:org1bf9464}

Using the notation for the problem in question, write out the known
variables and their values. The known variables will be a subset of
the following: \(p^i_x, p_x, p^i_y, p_y, v^i_x, v_x, v^i_y, v_y\) and
\(t\). The knowns should be written in the notation adopted for the
particular problem.

\subsubsection{Step 3: Identify Unknowns}
\label{sec:org52c043e}

Each problem will have at most 4 unknowns that need to be determined,
selected from the variables listed in the Step 2 that are not known.
The number of relevant unknowns will usually be less than 4, since
questions will often focus on one or two unknowns.  As an example, the
equation that horizontal velocity is constant is so trivial that most
problems will not look for this as an unknown.  The unknowns should be
written in the notation adopted for the particular problem.

\subsubsection{Step 4: Kinematic Equations}
\label{sec:org70a47f9}

Depending upon the known data and what is to be determined, a choice
should be made as to which four of the following five equations should
be applied between the two points on the path to obtain the most
direct solution to the problem.

\begin{enumerate}
\item Step 4.1: Horizontal Motion
\label{sec:orgdc76841}

From Equation \ref{Eq_horizVel}: \(v_x = v_x^i\) (The \textbf{velocity} in the
horizontal or \(x\) direction is \textbf{constant})

From Equation \ref{Eq_horizPos}: \(p_x = p_x^i + v_x^i t\)

\item Step 4.2: Vertical Motion
\label{sec:orgf44081b}

In the vertical or \(y\) direction \textbf{only two} of the following three
equations (using \(a_y = -g\)) can be used for solution.  (The sign of
\(g\) will change to positive if the positive \(y\) axis is downward.)

From Equation \ref{Eq_vertVel}: \(v_y = v_y^i - g t\)

From Equation \ref{Eq_vertPos}: \(p_y = p_y^i + v_y^i t - \frac{1}{2} g t^2\)

From Equation \ref{Eq_vertNoTime}: \(v_y^2 = (v_y^i)^2 - 2 g ( p_y - p_y^i)\)

For example, if the particle's final velocity \(v_y\) is not needed,
then the first and third of these questions (for \(y\)) will not be
useful.
\end{enumerate}

\subsubsection{Step 5: Solve for Unknowns}
\label{sec:org2d39b00}

Use the equations from Step 4, together with the known values from
Step 2 to find the unknown values from Step 3.  We can do this
systematically by going through each equation and determining how many
unknowns are in that equation.  Any equations with one unknown can be
used to solve for that unknown directly.

\section{Example (Sack Slides Off of Ramp)}
\label{sec:org5f089a8}

A sack slides off the ramp, shown in Figure \ref{fig:org28d7ddb}.  We can
ignore the physics of the sack sliding down the ramp and just focus on
its exit velocity from the ramp.  There is initially no vertical
component of velocity and the horizontal velocity is:

\begin{verbatim}
horiz_velo = 17 #m/s.
\end{verbatim}

\textbf{Task:} Determine the time needed for the sack to strike the floor and
the range \(R\) where sacks begin to pile up.

The acceleration due to gravity \(g\) is assumed to have the following
value.

\begin{verbatim}
g = 9.81 #m/s^2
\end{verbatim}

\begin{figure}[htbp]
\centering
\includegraphics[width=0.7\textwidth]{../SackExample.png}
\caption{\label{fig:org09b09e3}Coordinate System and Definition of Symbols}
\end{figure}

\subsection{Solution}
\label{sec:org373689d}

\textbf{Step 1: Coordinate System.} The origin of the coodinates is
established at the beginning of the path, point A (Figure
\ref{fig:org09b09e3}).  The initial positions and velocities will be taken
at Point \(A\) and the final positions and velocities will be taken at
Point \(B\).  Points \(A\) and \(B\) were selected because we know values at
the launch point (Point \(A\)) and we wish to find values related to the
time of flight and the landing point (Point \(B\)).

With respect to notation, we identify point \(A\) as the initial point
and point \(B\) as the final point.  Therefore, in equations
\ref{Eq_horizVelo},
\ref{Eq_horizPos},
\ref{Eq_vertVelo}, \ref{Eq_vertPos}
and \ref{Eq_vertNoTime} we have the following new
notation:

\(p^i_x = p^A_x, p_x = p^B_x, p^i_y = p^A_y, p_y = p^B_y, v^i_x = v^A_x, v_x = v^B_x, v^i_y = v^A_y, v_y = v^B_y\)

and \(t_{AB}\) refers to the time that passes when the particle moves from Point \(A\) to Point \(B\).

\textbf{Step 2: Identify Knowns.} We know values for 5 of the 9 possible
variables:

\begin{itemize}
\item \(p_x^A = 0\)
\item \(p_y^A = 0\)
\item \(v_x^A = \text{horiz_velo}\)
\item \(v_y^A = 0\)
\item \(p_y^B = -\text{height}\) (negative because below the origin)\$
\end{itemize}

\begin{verbatim}
pAx = 0
pAy = 0
vAx = horiz_velo
vAy = 0
pBy = -height
\end{verbatim}

\textbf{Step 3: Identify Unknowns.} According to the original question our goal is to find 2 unknowns:

\begin{itemize}
\item \(t_{AB}\): the time needed for the sack to strike the floor.
\item \(p_x^B\): the range \(R\), which is the x-coordinate of the final position.
\end{itemize}

(We also have 2 other unknowns that are not specifically asked for:
\(v_x^B\) and \(v_y^B\).  We may need to solve for these as part of the
solution for the requested unknown values.)

\textbf{Step 4: Kinematic Equations.}

\textbf{Step 4.1: Horizonal Motion.} From \ref{Eq_horizVel} we know:

\begin{equation}
\label{Eq_1}
v_x^B = v_x^A
\end{equation}

From \ref{Eq_horizPos} we know:

\begin{equation}
\label{Eq_2}
p_x^B = p_x^A + v_x^A t_{AB}
\end{equation}

\textbf{Step 4.2: Vertical Motion.} From \ref{Eq_vertVel} we know:

\begin{equation}
\label{Eq_3}
v_y^B = v_y^A - g t_{AB}
\end{equation}

From \ref{Eq_vertPos} we know:
\begin{equation}
\label{Eq_4}
p_y^B = p_y^A + v_y^A t_{AB} - \frac{1}{2} g t_{AB}^2
\end{equation}

From \ref{Eq_vertNoTime} we know:
\begin{equation}
\label{Eq_5}
(v_y^B)^2 = (v_y^A)^2 - 2 g ( p_y^B - p_y^A)
\end{equation}

\textbf{Step 5: Solve for the Unknowns.} We can go through each of the above
5 equations to see how many unknowns are in each equation:

\begin{itemize}
\item Eq\textsubscript{1} has one unknown: \(v_x^B\)
\item Eq\textsubscript{2} has two unknowns: \(p_x^B\), \(t_{AB}\)
\item Eq\textsubscript{3} has two unknowns: \(v_y^B\), \(t_{AB}\)
\item Eq\textsubscript{4} has one unknown: \(t_{AB}\)
\item Eq\textsubscript{5} has one unknown: \(v_y^B\)
\end{itemize}

From Step 3, we know that our goal is to find \(t_{AB}\) and \(p_x^B\).
We can see that \ref{Eq_4} allows us solve for \(t_{AB}\).  Once we have this
value, we can use \ref{Eq_2} to solve for \(p_x^B\).  (Although we could also
solve for \(v_x^B\) and \(v_y^B\), using \ref{Eq_1} and \ref{Eq_3} (or \ref{Eq_5}),
respectively, we are not asked to do so by the question.)

Find \(t_{AB}\) from \ref{Eq_4}:

\(p^B_y = p^A_y + v^A_y t - \frac{1}{2} g t_{AB}^2\)

Since \(v^A_y = 0\), 

\(p^B_y = p^A_y - \frac{1}{2} g t_{AB}^2\)

We can rearrange the above equation to solve for \(t_{AB}\): 

\(t_{AB} = \sqrt{(p^A_y - p^B_y)/\frac{1}{2}(g)}\)

The following code solves for \(t_{AB}\).

\begin{verbatim}
import math
tAB = math.sqrt((pAy - pBy)/(0.5*(g)))
print("ANSWER tAB = ", tAB, "s")
\end{verbatim}
\end{document}