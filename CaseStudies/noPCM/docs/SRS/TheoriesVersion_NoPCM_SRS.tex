% !TEX program = lualatex
\documentclass[12pt]{article}
\usepackage{fontspec}
\usepackage{fullpage}
\usepackage{hyperref}
\hypersetup{bookmarks=true,colorlinks=true,linkcolor=red,citecolor=blue,filecolor=magenta,urlcolor=cyan}
\usepackage{amsmath}
\usepackage{amssymb}
\usepackage{mathtools}
\usepackage{unicode-math}
\usepackage{tabu}
\usepackage{longtable}
\usepackage{booktabs}
\usepackage{caption}
\usepackage{graphics}
\usepackage{svg}
\usepackage{enumitem}
\usepackage{filecontents}
\usepackage[backend=bibtex]{biblatex}
\usepackage{url}
\usepackage{siunitx}

\usepackage{color}

\newif\ifcomments\commentstrue

\ifcomments
\newcommand{\authornote}[3]{\textcolor{#1}{[#3 ---#2]}}
\newcommand{\todo}[1]{\textcolor{red}{[TODO: #1]}}
\else
\newcommand{\authornote}[3]{}
\newcommand{\todo}[1]{}
\fi

\newcommand{\wss}[1]{\authornote{blue}{SS}{#1}} 
\newcommand{\jc}[1]{\authornote{magenta}{JC}{#1}} %For explanation of the template
\newcommand{\jb}[1]{\authornote{cyan}{JB}{#1}}

\setmathfont{Latin Modern Math}
\newcommand{\gt}{\ensuremath >}
\newcommand{\lt}{\ensuremath <}
\global\tabulinesep=1mm
\newlist{symbDescription}{description}{1}
\setlist[symbDescription]{noitemsep, topsep=0pt, parsep=0pt, partopsep=0pt}
\bibliography{bibfile}
\title{Software Requirements Specification for Solar Water Heating Systems}
\author{Thulasi Jegatheesan}

\begin{document}

Recently we have determined that the exiting definitions for theoretical models,
data definitions, general definitions and instance models are not entirely
consistent.  Moreover, the terminology, especially the terminology that combines
``theory'' and ``model'' is confusing.

Our new terminology will be based on the concept of different ``levels'' of
theories.  We currently have what we are calling context theories, background
theories, theories and final theories.  Context theories are the starting point
for a given scientific model.  These theories are not defined, but their
implicit presence is necessary for completeness.  Typical examples include
theories for arithmetic operations, differentiation, integration, vector
calculus, etc.  Following the context theories we have background theories.
Background theories are defined, but not derived.  Typically they will be the
general forms of the conservation equations, like conservation of thermal energy
or momentum, and constitutive equations.  Assumptions are often invoked by The
level of detail used to define background theories can vary, since we do not
always need the full theory; we sometimes just need to know that the theory
exists.  The background theories are refined into other theories by making
assumptions, like plane stress, or linear elasticity, or isothermal material
properties, or laminar flow, etc.  These theories are combined and refined until
the point where we have final theories. All theories are part of the
documentation, but the final theories are the ones that will be transformed into
code.

We also have data definitions.  A data definition is a label for part of a
theory. Assumptions will be maintained from the previous terminology.

Below is an attempt to clarify the new concept of different levels of theories
by re-writing the relevant parts of the NoPCM model using the new conceptual
model of theories.

%%%%%%%%%%%%%%%%%%%%%%%%%%%%%%%%%%%%%%%%%%%%%%%%%%%%%%%%%%%%%%%%%%%%%%%%%%%%%%%%%%

\subsection{Terminology Definition}

\begin{itemize}
\item \textit{thermal flux} {Provide the sign convention for thermal flux.}
\end{itemize}

%%%%%%%%%%%%%%%%%%%%%%%%%%%%%%%%%%%%%%%%%%%%%%%%%%%%%%%%%%%%%%%%%%%%%%%%%%%%%%%%%%

\section{Assumptions} \label{Sec:Assumps}

\subsection{Assumptions for Background and Refined Theories}

\begin{itemize}
\item[Heat-Transfer-Coeffs-Constant:\phantomsection\label{assumpHTCC}]{All heat transfer coefficients are constant over time. (RefBy: \hyperref[BT:nwtnCooling]{BT:nwtnCooling}.)}
\item[Constant-Water-Temp-Across-Tank:\phantomsection\label{assumpCWTAT}]{The water in the tank is fully mixed, so the temperature of the water is the same throughout the entire tank. (RefBy: \hyperref[RT:rocTempSimp]{RT:rocTempSimp}.)}
\item[Density-Water-Constant-over-Volume:\phantomsection\label{assumpDWCoW}]{The density of water has no spatial variation; that is, it is constant over their entire volume. (RefBy: \hyperref[RT:rocTempSimp]{RT:rocTempSimp}.)}
\item[Specific-Heat-Energy-Constant-over-Volume:\phantomsection\label{assumpSHECoW}]{The specific heat capacity of water has no spatial variation; that is, it is constant over its entire volume. (RefBy: \hyperref[RT:rocTempSimp]{RT:rocTempSimp}.)}
\item[Newton-Law-Convective-Cooling-Coil-Water:\phantomsection\label{assumpLCCCW}]{Newton's law of convective cooling applies between the heating coil and the water. (RefBy: \hyperref[RT:htFluxWaterFromCoil]{RT:htFluxWaterFromCoil}.)}
\item[Temp-Heating-Coil-Constant-over-Time:\phantomsection\label{assumpTHCCoT}]{The temperature of the heating coil is constant over time. (RefBy: \hyperref[likeChgTCVOD]{LC:Temperature-Coil-Variable-Over-Day} and \hyperref[RT:htFluxWaterFromCoil]{RT:htFluxWaterFromCoil}.)}
\item[Charging-Tank-No-Temp-Discharge:\phantomsection\label{assumpCTNTD}]{The model only accounts for charging the tank, not discharging. The temperature of the water can only increase, or remain constant; it cannot decrease. This implies that the initial temperature is less than (or equal to) the temperature of the heating coil. (RefBy: \hyperref[likeChgDT]{LC:Discharging-Tank}.)}
\item[Water-Always-Liquid:\phantomsection\label{assumpWAL}]{The operating temperature range of the system is such that the material (water in this case) is always in liquid state. That is, the temperature will not drop below the melting point temperature of water, or rise above its boiling point temperature. (RefBy: \hyperref[unlikeChgWFS]{UC:Water-Fixed-States}, \hyperref[BT:sensHtE]{BT:sensHtE}, \hyperref[FT:heatEInWtr]{FT:heatEInWtr}, and \hyperref[FT:eBalanceOnWtr]{FT:eBalanceOnWtr}.)}

\item[Atmospheric-Pressure-Tank:\phantomsection\label{assumpAPT}]{The pressure in the tank is atmospheric, so the melting point temperature and boiling point temperature of water are 0${{}^{\circ}\text{C}}$ and 100${{}^{\circ}\text{C}}$, respectively. (RefBy: \hyperref[FT:heatEInWtr]{FT:heatEInWtr}.)}
\item[Volume-Coil-Negligible:\phantomsection\label{assumpVCN}]{When considering the volume of water in the tank, the volume of the heating coil is assumed to be negligible. (RefBy: \hyperref[DD:waterVolume.nopcm]{DD:waterVolume\_nopcm}.)}

\item[\wss{New:}] The assumptions from this point on are newly added.
 
\item[UniformDensity-OverVol:\phantomsection\label{assumpUnifDens}]{The density
of the liquid does not depend on the spatial location within the entire volume.
($\forall \mathbf{s}, \mathbf{s'} | \rho(\mathbf{s}, T) = \rho(\mathbf{s'}, T)$)
(RefBy: \hyperref[RT:consThermE:ReduceDep]{RT:consThermE:ReduceDep}.)}
\item[UniformSpecHeat-OverVol:\phantomsection\label{assumpUnifSpecHeat}]{The
specific heat of the liquid does not depend on the spatial location within the
entire volume. (RefBy: \hyperref[RT:consThermE:ReduceDep]{RT:consThermE:ReduceDep}.)}
\item[UniformTemp-OverVol:\phantomsection\label{assumpUnifTemp}]{The
specific heat of the liquid does not depend on the spatial location within the
entire volume. (RefBy: \hyperref[RT:consThermE:ReduceDep]{RT:consThermE:ReduceDep}.)}
\item[UniformHeatGen-OverVol:\phantomsection\label{assumpUnifHeatGen}]{The
heat generation of the liquid does not depend on the spatial location within the
entire volume. (RefBy: \hyperref[RT:consThermE:ReduceDep]{RT:consThermE:ReduceDep}.)}
\item[DensityIndepTemp:\phantomsection\label{assumpDensIndepT}]{The density
of the material does not depend on the temperature. (RefBy: \hyperref[RT:consThermE:ReduceDep]{RT:consThermE:ReduceDep}.)}
\item[SpecHeatIndepTemp:\phantomsection\label{assumpSpecHeatIndepT}]{The
specific heat capacity of the material does not depend on the temperature.
(RefBy: \hyperref[RT:consThermE:ReduceDep]{RT:consThermE:ReduceDep}.)}
\item[VolumeIsCompact:\phantomsection\label{assumpVolCompact}]{A compact volume
has the property of compactness, which ``generalizes the notion of a subset of
Euclidean space being closed (containing all its limit points) and bounded
(having all its points lie within some fixed distance of each other)''
\cite{Wiki-CompactSpace2022}. (RefBy:
\hyperref[RT:rocTempSimp]{RT:rocTempSimp}.)}
\item[VolHasPiecewise-Smooth-Surface:\phantomsection\label{assumpPiecewiseSmooth}]
{The surface of the volume can be broken into separate surfaces which are all
smooth, where the surface is smooth if its partial derivatives of every order
exist at every point of the domain \cite{Wiki-DiffGeom2022}. (RefBy:
\hyperref[RT:rocTempSimp]{RT:rocTempSimp}.)}
\item[ThermalFlux-UniformOverSurfaces:\phantomsection\label{assumpUnifThermFlux}]
{The projection of the thermal flux in the direction of the unit outward normal
is uniform over each surface that encloses the volume (RefBy:
\hyperref[RT:rocTempSimp]{RT:rocTempSimp}.)}
\item[UniformHeatTransCoeffOverSurf:\phantomsection\label{assumpUnifHeatTransCoeff}]
{The heat transfer coefficient over the surface is uniform.  That is, the heat
transfer coefficient does not depend on the location on the surface (RefBy:
\hyperref[RT:nwtnCooling:ReduceDep]{RT:nwtnCooling:ReduceDep}.)}
\item[UniformTempOverSurf:\phantomsection\label{assumpUnifTempSurf}]
{The temperature at the surface is uniform.  That is, the temperature does not
depend on the location on the surface (RefBy:
\hyperref[RT:nwtnCooling:ReduceDep] {RT:nwtnCooling:ReduceDep}.)}
\item[UniformEnviroTemp:\phantomsection\label{assumpUnifEnviroTempSurf}]
{The temperature at the environment around the surface is uniform.  That is, the
environment temperature does not depend on the location on the surface (RefBy:
\hyperref[RT:nwtnCooling:ReduceDep] {RT:nwtnCooling:ReduceDep}.)}
\item[HeatTransCoeffIndepTime:\phantomsection\label{assumpHeatTransIndepTime}]
{The heat transfer coefficient does not change over time. (RefBy:
\hyperref[RT:nwtnCooling:ReduceDep] {RT:nwtnCooling:ReduceDep}.)}
\item[HeatTransCoeffIndepTemp:\phantomsection\label{assumpHeatTransIndepTemp}]
{The heat transfer coefficient is independent of temperature. (RefBy:
\hyperref[RT:nwtnCooling:ReduceDep] {RT:nwtnCooling:ReduceDep}.)}

\end{itemize}

\subsection{Assumptions for Final Theories}

\begin{itemize}

\item[Thermal-Energy-Only:\phantomsection\label{assumpTEO}]{The only form of
energy that is relevant for this problem is thermal energy. All other forms of
energy, such as mechanical energy, are assumed to be negligible. (RefBy:
\hyperref[BT:consThermE]{BT:consThermE}.)}
                
\item[UniformFullyMixed:\phantomsection\label{assumpFullyMixed}]
{The water tank is fully mixed so that the material properties and temperature
are uniform throughout its volume. (RefBy:
\hyperref[RT:htFluxWaterFromCoil]{RT:htFluxWaterFromCoil}.)}

\item[No-Internal-Heat-Generation-By-Water:\phantomsection\label{assumpNIHGBW}]{No
internal heat is generated by the water; therefore, the volumetric heat
generation per unit volume is zero. (RefBy:
\hyperref[unlikeChgNIHG]{UC:No-Internal-Heat-Generation} and
\hyperref[FT:eBalanceOnWtr]{FT:eBalanceOnWtr}.)}

\item[Perfect-Insulation-Tank:\phantomsection\label{assumpPIT}]{The tank is
perfectly insulated so that there is no heat loss from the tank. (RefBy:
\hyperref[likeChgTLH]{LC:Tank-Lose-Heat} and
\hyperref[FT:eBalanceOnWtr]{FT:eBalanceOnWtr}.)}

\item[DensityIndepTempWater:\phantomsection\label{assumpDensIndepTWater}]{The
density of the water does not depend on the temperature. (RefBy:
\hyperref[FT:eBalanceOnWtr]{FT:eBalanceOnWtr}.)}

\item[SpecHeatIndepTempWater:\phantomsection\label{assumpSpecHeatIndepTWater}]{The
specific heat capacity of the material does not depend on the temperature.
(RefBy: \hyperref[RT:consThermE:ReduceDep]{RT:consThermE:ReduceDep}.)}

\item[UniformHeatTransCoeffOverCoil:\phantomsection\label{assumpUnifHeatTransCoeffCoil}]
{The heat transfer coefficient over the surface of the coil is uniform.  That
is, the heat transfer coefficient does not depend on the location on the coil's
surface (RefBy: \hyperref[RT:htFluxWaterFromCoil]{RT:htFluxWaterFromCoil}.)}

\item[Temp-Heating-Coil-Constant-over-Length:\phantomsection\label{assumpTHCCoL}]{The
temperature of the heating coil does not vary along its length. (RefBy:
\hyperref[likeChgTCVOL]{LC:Temperature-Coil-Variable-Over-Length},
\hyperref[RT:htFluxWaterFromCoil]{RT:htFluxWaterFromCoil}.)}

\item[HeatTransCoeffCoilIndepTime:\phantomsection\label{assumpHeatTransCoilIndepTime}]
{The heat transfer coefficient between the coil and the tank of water does not
change over time. (RefBy:
\hyperref[RT:htFluxWaterFromCoil]{RT:htFluxWaterFromCoil}.)}

\item[HeatTransCoeffCoilIndepTemp:\phantomsection\label{assumpHeatTransCoilIndepTemp}]
{The heat transfer coefficient between the coil and the tank of water does not
depend on the temperature. (RefBy:
\hyperref[RT:htFluxWaterFromCoil]{RT:htFluxWaterFromCoil}.)}

\end{itemize}

%%%%%%%%%%%%%%%%%%%%%%%%%%%%%%%%%%%%%%%%%%%%%%%%%%%%%%%%%%%%%%%%%%%%%%%%%%%%%%%%%%

\section{Context Theories}

Some theories do not have to be explicitly invoked.  They are part of the
context for the other theories, without having to be explicitly stated or
defined.  The context theories for this problem are as follows: 

\begin{itemize}
  \item arithmetic
  \item operations 
  \item differentiation
  \item partial differentiation
  \item integration
  \item vector calculus (gradient operator, dot product)
  \item Gauss's divergence theorem \wss{Should this be a separate theory?
  Possibly a theory that is in the background, but not actually printed in the
  SRS document?}
\end{itemize}

%%%%%%%%%%%%%%%%%%%%%%%%%%%%%%%%%%%%%%%%%%%%%%%%%%%%%%%%%%%%%%%%%%%%%%%%%%%%%%%%%%

\section{Background Theories (BT)} \label{Sec:BTs}

%%%%%%%%%%%%%%%%%%%%%%%%%%%%%%%%%%%%%%

%\vspace{\baselineskip}
\noindent
\begin{minipage}{\textwidth}
\begin{tabular}{>{\raggedright}p{0.13\textwidth}>{\raggedright\arraybackslash}p{0.82\textwidth}}
\toprule \textbf{Refname} & \textbf{BT:consThermE}
\phantomsection 
\label{BT:consThermE}
\\ \midrule \\
Label & Conservation of thermal energy
        
\\ \midrule \\
Equation & \begin{displaymath}
        -{\nabla \cdot \mathbf{q} (\mathbf{x}, t)} + g(\mathbf{x}, t) = \rho (\mathbf{x}, T) C (\mathbf{x}, T) \frac{\partial T(\mathbf{x}, t)}{\partial t}
           \end{displaymath}
\\ \midrule \\
Description & \begin{symbDescription}
              \item{$∇$ is the gradient (Unitless)}
              \item{$\symbf{q}$ is the thermal flux vector ($\frac{\text{W}}{\text{m}^{2}}$)}
              \item{$g$ is the volumetric heat generation per unit volume ($\frac{\text{W}}{\text{m}^{3}}$)}
              \item{$ρ$ is the density ($\frac{\text{kg}}{\text{m}^{3}}$)}
              \item{$C$ is the specific heat capacity ($\frac{\text{J}}{\text{kg}{}^{\circ}\text{C}}$)}
              \item{$t$ is the time (${\text{s}}$)}
              \item{$T$ is the temperature (${{}^{\circ}\text{C}}$)}
              \item \wss{$\mathbf{x}$ is the spatial position (${\text{m}}$)}
              \end{symbDescription}
\\ \midrule \\
Notes & The above equation gives the law of conservation of energy for transient
        heat transfer in a given material.  For this equation to apply, other
        forms of energy, such as mechanical energy, are assumed to be negligible
        in the system (\hyperref[assumpTEO]{A:Thermal-Energy-Only}). \wss{This
        form of the conservation equation is general.  A specific coordinate
        system (Cartesian, polar, spherical etc) is not assumed.  The spatial
        location is represented sympolically by $\mathbf{x}$.  The divergence
        operator ($∇ \cdot$) can be defined for whatever coordinate system is
        used.  The thermal flux ($\symbf{q}$), the volumetric heat generation
        ($g$) and temperature ($T$) depend on spatial postion ($\mathbf{x}$) and
        time ($t$).  The material properties, density ($ρ$) and specific heat
        capacity ($C$), depend on spatial postion ($\mathbf{x}$) and temperature
        ($T$).}
\\ \midrule \\
Source & \hyperref{http://www.efunda.com/formulae/heat_transfer/conduction/overview_cond.cfm}{}{}{Fourier Law of Heat Conduction and Heat Equation}
\\ \midrule \\
RefBy & \hyperref[RT:rocTempSimp]{RT:rocTempSimp},
\hyperref[RT:consThermE:ReduceDep]{RT:consThermE:ReduceDep}
        
\\ \bottomrule
\end{tabular}
\end{minipage}

%%%%%%%%%%%%%%%%%%%%%%%%%%%%%%%%%%%%%%

\paragraph{Preconditions for \hyperref[BT:consThermE]{BT:consThermE}:}
\label{BT:consThermEPrecond}

\begin{itemize}
\item \hyperref[assumpTEO]{A:Thermal-Energy-Only}
\end{itemize}

%%%%%%%%%%%%%%%%%%%%%%%%%%%%%%%%%%%%%%

\vspace{\baselineskip}
\noindent
\begin{minipage}{\textwidth}
\begin{tabular}{>{\raggedright}p{0.13\textwidth}>{\raggedright\arraybackslash}p{0.82\textwidth}}
\toprule \textbf{Refname} & \textbf{BT:sensHtE}
\phantomsection 
\label{BT:sensHtE}
\\ \midrule \\
Label & Sensible heat energy (no state change)
        
\\ \midrule \\
Equation & \begin{displaymath}
           E={C^{\text{L}}} m ΔT
           \end{displaymath}
\\ \midrule \\
Description & \begin{symbDescription}
              \item{$E$ is the sensible heat (${\text{J}}$)}
              \item{${C^{\text{L}}}$ is the specific heat capacity of a liquid ($\frac{\text{J}}{\text{kg}{}^{\circ}\text{C}}$)}
              \item{$m$ is the mass (${\text{kg}}$)}
              \item{$ΔT$ is the change in temperature (${{}^{\circ}\text{C}}$)}
              \end{symbDescription}
\\ \midrule \\
Notes & $E$ occurs as long as the material does not reach a temperature where a
phase change occurs, as assumed in \hyperref[assumpWAL]{A:Water-Always-Liquid}.
\wss{This could actually be an assumption of no state change.  This could be a generic assumption that applies for any state of matter.  This particular instantiation applies for liquid, but with a change in the symbol for specific heat capacity it could be changed to a solid (for instance).}
        
\\ \midrule \\
Source & \hyperref{http://en.wikipedia.org/wiki/Sensible_heat}{}{}{Definition of Sensible Heat}
         
\\ \midrule \\
RefBy & \hyperref[FT:heatEInWtr]{FT:heatEInWtr}

\\ \bottomrule
\end{tabular}
\end{minipage}

%%%%%%%%%%%%%%%%%%%%%%%%%%%%%%%%%%%%%%

\vspace{\baselineskip}
\noindent
\begin{minipage}{\textwidth}
\begin{tabular}{>{\raggedright}p{0.13\textwidth}>{\raggedright\arraybackslash}p{0.82\textwidth}}
\toprule \textbf{Refname} & \textbf{BT:nwtnCooling}
\phantomsection 
\label{BT:nwtnCooling}
\\ \midrule \\
Label & Newton's law of cooling
        
\\ \midrule \\
Equation & \begin{displaymath} 
        \textbf{q}\left(\textbf{s}, t\right) \cdot
        \symbf{\hat{n}} = h(\textbf{s}, t, T) (T(\textbf{s}, t) -
        T_{\text{env}}(\textbf{s}, t))
        \end{displaymath}
\\ \midrule \\
Description & \begin{symbDescription}
              \item{$\textbf{q}$ is the thermal flux vector ($\frac{\text{W}}{\text{m}^{2}}$)}
              \item \wss{$\symbf{\hat{n}}$ is a unit outward normal from a surface (${\text{m}}$)}
              \item \wss{$\textbf{s}$ is a location on the surface (${\text{m}}$)}
              \item{$t$ is the time (${\text{s}}$)}
              \item{$h$ is the convective heat transfer coefficient ($\frac{\text{W}}{\text{m}^{2}{}^{\circ}\text{C}}$)}
              \item{$T$ is the temperature of the body (${{}^{\circ}\text{C}}$)}
              \item{$T_{\text{env}}$ is the temperature of the environment surrounding the body (${{}^{\circ}\text{C}}$)}
              \end{symbDescription}
\\ \midrule \\
Notes & Newton's law of cooling describes convective cooling from a surface. The
law is stated as: the rate of heat loss from a body is proportional to the
difference in temperatures between the body and its surroundings.  In the form
given the thermal flux ($\textbf{q}$), heat transfer coefficient ($h$),
temperature of the body ($T$) and temperature of the environment
($T_\text{env}$) can all vary with the location on the surface ($\textbf{s}$)
and over time ($t$).

\\ \midrule \\
Source & \cite[(pg. 8)]{incroperaEtAl2007}
         
\\ \midrule \\
RefBy & \hyperref[RT:htFluxWaterFromCoil]{RT:htFluxWaterFromCoil}
        
\\ \bottomrule
\end{tabular}
\end{minipage}

%%%%%%%%%%%%%%%%%%%%%%%%%%%%%%%%%%%%%%%%%%%%%%%%%%%%%%%%%%%%%%%%%%%%%%%%%%%%%%%%%%

\vspace{\baselineskip}
\noindent
\begin{minipage}{\textwidth}
\begin{tabular}{>{\raggedright}p{0.13\textwidth}>{\raggedright\arraybackslash}p{0.82\textwidth}}
\toprule \textbf{Refname} & \textbf{BT:density}
\phantomsection 
\label{BT:density}
\\ \midrule \\
Label & Density
        
\\ \midrule \\
Equation & \begin{displaymath}
           \rho = \frac {m} {V}
           \end{displaymath}
\\ \midrule \\
Description & \begin{symbDescription}
              \item{$\rho$ is the density of a material ($\frac{\text{kg}}{\text{m}^{3}}$)}
              \item{$m$ is the mass of the body (${\text{kg}}$)}
              \item{$V$ is the volume of the body (${\text{m}^{3}}$)}
              \end{symbDescription}
\\ \midrule \\
Notes & Density is the mass per unit volume.
        
\\ \midrule \\
Source & --
         
\\ \midrule \\
RefBy & \hyperref[BT:consThermE]{BT:consThermE},
\hyperref[RT:rocTempSimp]{RT:rocTempSimp}, \hyperref[DD:waterMass]{DD:waterMass}
        
\\ \bottomrule
\end{tabular}
\end{minipage}

%%%%%%%%%%%%%%%%%%%%%%%%%%%%%%%%%%%%%%%%%%%%%%%%%%%%%%%%%%%%%%%%%%%%%%%%%%%%%%%%%%

\section{Refined Theories} \label{Sec:RefndTheories}

This section collects the laws and equations that will be used to build the instance models.

%%%%%%%%%%%%%%%%%%%%%%%%%%%%%%%%%%%%%%

\vspace{\baselineskip}
\noindent
\begin{minipage}{\textwidth}
\begin{tabular}{>{\raggedright}p{0.13\textwidth}>{\raggedright\arraybackslash}p{0.82\textwidth}}
\toprule \textbf{Refname} & \textbf{RT:consThermE:ReduceDep}
\phantomsection 
\label{RT:consThermE:ReduceDep}
\\ \midrule \\
Label & Refine conservation of thermal energy to simplify the equation by reducing the dependencies of the variables on space and temperature.
        
\\ \midrule \\
Equation & \begin{displaymath}
        -{\nabla \cdot \mathbf{q} (\mathbf{x}, t)} + g(t) = \rho C \frac{dT(t)}{dt}
        \end{displaymath}
\\ \midrule \\
Description & \begin{symbDescription}
        \item{$∇$ is the gradient (Unitless)}
        \item{$\symbf{q}$ is the thermal flux vector ($\frac{\text{W}}{\text{m}^{2}}$)}
        \item{$g$ is the volumetric heat generation per unit volume ($\frac{\text{W}}{\text{m}^{3}}$)}
        \item{$ρ$ is the density ($\frac{\text{kg}}{\text{m}^{3}}$)}
        \item{$C$ is the specific heat capacity ($\frac{\text{J}}{\text{kg}{}^{\circ}\text{C}}$)}
        \item{$t$ is the time (${\text{s}}$)}
        \item{$T$ is the temperature (${{}^{\circ}\text{C}}$)}
        \item \wss{$\mathbf{x}$ is the spatial position}
        \end{symbDescription}
\\ \midrule \\
Source & -- 
\\ \midrule \\
         
RefBy & \hyperref[RT:rocTempSimp]{RT:rocTempSimp}

\\ \bottomrule
\end{tabular}
\end{minipage}

%%%%%%%%%%%%%%%%%%%%%%%%%%%%%%%%%%%%%%

\paragraph{Preconditions for \hyperref[RT:consThermE:ReduceDep]{RT:consThermE:ReduceDep}:}
\label{RT:consThermE:ReduceDepPrecond}

\begin{itemize}
\item \hyperref[assumpTEO]{A:Thermal-Energy-Only}
\item \hyperref[assumpUnifDens]{A:UniformDensity-OverVol},
\item \hyperref[assumpUnifSpecHeat]{A:UniformSpecHeat-OverVol},
\item \hyperref[assumpUnifTemp]{A:UniformTemp-OverVol},
\item \hyperref[assumpUnifHeatGen]{A:UniformHeatGen-OverVol},
\item \hyperref[assumpDensIndepT]{A:DensityIndepTemp},
\item \hyperref[assumpSpecHeatIndepT]{A:SpecHeatIndepTemp} 
\end{itemize}

%%%%%%%%%%%%%%%%%%%%%%%%%%%%%%%%%%%%%%

\paragraph{Detailed derivation of constant material properties conservation equation:}
\label{RT:consThermE:ReduceDep:Deriv}

Starting from \hyperref[BT:consThermE]{BT:consThermE}:

\begin{displaymath}
-{\nabla \cdot \mathbf{q} (\mathbf{x}, t)} + g(\mathbf{x}, t) = \rho (\mathbf{x}, T) C (\mathbf{x}, T) \frac{\partial T(\mathbf{x}, t)}{\partial t}
\end{displaymath}

\noindent Apply the assumption that the density does not depend on the spatial location
(\hyperref[assumpUnifDens]{A:UniformDensity-OverVol}):

\begin{displaymath}
-{\nabla \cdot \mathbf{q} (\mathbf{x}, t)} + g(\mathbf{x}, t) = \rho (T) C (\mathbf{x}, T) \frac{\partial T(\mathbf{x}, t)}{\partial t}
\end{displaymath}

\noindent Apply the assumption that the specific heat does not depend on the spatial
location (\hyperref[assumpUnifSpecHeat]{A:UniformSpecHeat-OverVol}):

\begin{displaymath}
-{\nabla \cdot \mathbf{q} (\mathbf{x}, t)} + g(\mathbf{x}, t) = \rho (T) C (T) \frac{\partial T(\mathbf{x}, t)}{\partial t}
\end{displaymath}

\noindent Apply the assumption that the temperature does not depend on the spatial
location (\hyperref[assumpUnifTemp]{A:UniformTemp-OverVol}):

\begin{displaymath}
-{\nabla \cdot \mathbf{q} (\mathbf{x}, t)} + g(\mathbf{x}, t) = \rho (T) C (T) \frac{dT(t)}{dt}
\end{displaymath}

\noindent Apply the assumption that the heat generation does not depend on the spatial
location (\hyperref[assumpUnifHeatGen]{A:UniformHeatGen-OverVol}):

\begin{displaymath}
-{\nabla \cdot \mathbf{q} (\mathbf{x}, t)} + g(t) = \rho (T) C (T) \frac{dT(t)}{dt}
\end{displaymath}

\noindent Apply the assumption that the density is independent of the temperature
(\hyperref[assumpDensIndepT]{A:DensityIndepTemp}):

\begin{displaymath}
-{\nabla \cdot \mathbf{q} (\mathbf{x}, t)} + g(t) = \rho C (T) \frac{dT(t)}{dt}
\end{displaymath}

\noindent Apply the assumption that the specific heat is independent of the temperature
(\hyperref[assumpSpecHeatIndepT]{A:SpecHeatIndepTemp}):

\begin{displaymath}
-{\nabla \cdot \mathbf{q} (\mathbf{x}, t)} + g(t) = \rho C \frac{dT(t)}{dt}
\end{displaymath}

%%%%%%%%%%%%%%%%%%%%%%%%%%%%%%%%%%%%%%

\vspace{\baselineskip}
\noindent
\begin{minipage}{\textwidth}
\begin{tabular}{>{\raggedright}p{0.13\textwidth}>{\raggedright\arraybackslash}p{0.82\textwidth}}
\toprule \textbf{Refname} & \textbf{RT:rocTempSimp}
\phantomsection 
\label{RT:rocTempSimp}
\\ \midrule \\
Label & Simplified rate of change of temperature
        
\\ \midrule \\
Equation & \begin{displaymath}
           m C \frac{\,dT(t)}{\,dt} = -\sum_{i=0}^n {q_i(t)} {A_i} + g(t) V
           \end{displaymath}
\\ \midrule \\
Description & \begin{symbDescription}
              \item{$m$ is the mass (${\text{kg}}$)}
              \item{$C$ is the specific heat capacity ($\frac{\text{J}}{\text{kg}{}^{\circ}\text{C}}$)}
              \item{$t$ is the time (${\text{s}}$)}
              \item{$T$ is the temperature (${{}^{\circ}\text{C}}$)}
              \item{${q_{i}}$ is the heat flux input across surface $i$ ($\frac{\text{W}}{\text{m}^{2}}$)} \wss{changed}
              \item{${A_{i}}$ is the surface area for surface $i$ over which heat is transferred in (${\text{m}^{2}}$)} \wss{changed}
              \item{$g$ is the volumetric heat generation per unit volume ($\frac{\text{W}}{\text{m}^{3}}$)}
              \item{$V$ is the volume (${\text{m}^{3}}$)}
              \item \wss{$n$ the number of surfaces over which heat transfer occurs}
              \end{symbDescription}
\\ \midrule \\
Source & --
         
\\ \midrule \\
RefBy & \hyperref[FT:eBalanceOnWtr]{FT:eBalanceOnWtr}

\\ \bottomrule
\end{tabular}
\end{minipage}

%%%%%%%%%%%%%%%%%%%%%%%%%%%%%%%%%%%%%%

\paragraph{Preconditions for \hyperref[RT:rocTempSimp]{RT:rocTempSimp}:}
\label{RT:rocTempSimpPrecond}

\begin{itemize}
\item \hyperref[assumpTEO]{A:Thermal-Energy-Only},
\item \hyperref[assumpUnifDens]{A:UniformDensity-OverVol},
\item \hyperref[assumpUnifSpecHeat]{A:UniformSpecHeat-OverVol},
\item \hyperref[assumpUnifTemp]{A:UniformTemp-OverVol},
\item \hyperref[assumpUnifHeatGen]{A:UniformHeatGen-OverVol},
\item \hyperref[assumpDensIndepT]{A:DensityIndepTemp},
\item \hyperref[assumpSpecHeatIndepT]{A:SpecHeatIndepTemp},
\item \hyperref[assumpVolCompact]{A:VolumeIsCompact},
\item \hyperref[assumpPiecewiseSmooth]{A:VolHasPiecewise-Smooth-Surface},
\item \hyperref[assumpUnifThermFlux]{A:ThermalFlux-UniformOverSurfaces}
\end{itemize}

%%%%%%%%%%%%%%%%%%%%%%%%%%%%%%%%%%%%%%

\paragraph{Detailed derivation of simplified rate of change of temperature:}
\label{RT:rocTempSimpDeriv}

Integrating \hyperref[RT:consThermE:ReduceDep]{RT:consThermE:ReduceDep} over a
volume ($V$), we have:

\begin{displaymath}
-\int_{V}{∇\cdot{}\symbf{q}(\textbf{x}, t)}\,dV+\int_{V}{g(t)}\,dV=\int_{V}{ρ C \frac{\,dT(t)}{\,dt}}\,dV
\end{displaymath}

Moving all terms that do not vary over space outside of the volume integrals and
integrating the remaining volume integrals we have:

\begin{displaymath}
-\int_{V}{∇\cdot{}\symbf{q}(\textbf{x}, t)}\,dV + g(t) V=ρ C V \frac{dT(t)}{dt}
\end{displaymath}

Assuming the volume is compact
(\hyperref[assumpVolCompact]{A:VolumeIsCompact}) and has a piecewise smooth
surface (\hyperref[assumpPiecewiseSmooth]{A:VolHasPiecewise-Smooth-Surface}), we
can apply Gauss's Divergence Theorem to the first term over the surface $S$ of
the volume, with $\symbf{q}$ as the thermal flux vector from the surface,
$\textbf{s}$ as the location on the surface, and $\symbf{\hat{n}}$ as a unit
outward normal vector from the surface:

\begin{displaymath}
-\int_{S}{\symbf{q}(\textbf{s}, t)\cdot{}\symbf{\hat{n}}}\,dS+ g(t) V=ρ C V \frac{dT(t)}{dt}
\end{displaymath}
        
The surface $S$ of the volume $V$ can be divided into $n$ separate surfaces
$S_i$, each of area $A_i$.  For each of these surfaces $S_i$ we assume $q_i (t)
= \mathbf{q}(\mathbf{s}, t) \cdot \symbf{\hat{n}}$ is uniform. That is, the
projection of the thermal flux $q_i$ in the direction of the unit outward normal
is uniform over the surface $S_i$
(\hyperref[assumpUnifThermFlux]{A:ThermalFlux-UniformOverSurfaces}).  Since the
normal thermal flux does not vary over the surface, it can be moved outside of
the integration.  The integral over the surface then is simply the surface area
$A_i$. With this information the previous equation can be rewritten as:

\begin{displaymath}
-\sum_{i=0}^n {q_i(t)} {A_i} + g(t) V = ρ C V \frac{dT(t)}{dt}
\end{displaymath}

Substituting in $ρ$=$m$/$V$ (\hyperref[BT:density]{BT:density}), the above
equation can be rewritten as:

\begin{displaymath}
m C \frac{\,dT(t)}{\,dt}= -\sum_{i=0}^n {q_i(t)} {A_i} + g(t) V
\end{displaymath}

When applying the above equation, remember that the sign of $q_i$ is determined
relative to the unit outward normal each surface.  If the thermal flux is
travelling out of the body, in the direction of the unit outward normal, then
the sign of $q_i$ is positive.  If the thermal flux is travelling into the body,
then the direction is opposite to the unit outward normal and the sign will be
negative.

%%%%%%%%%%%%%%%%%%%%%%%%%%%%%%%%%%%%%%

\vspace{\baselineskip}
\noindent
\begin{minipage}{\textwidth}
\begin{tabular}{>{\raggedright}p{0.13\textwidth}>{\raggedright\arraybackslash}p{0.82\textwidth}}
\toprule \textbf{Refname} & \textbf{RT:nwtnCooling:ReduceDep}
\phantomsection 
\label{RT:nwtnCooling:ReduceDep}
\\ \midrule \\
Label & Refine Newton's law of cooling to simplify the relation by reducing the
dependencies of the variables on location and temperature. 
\\ \midrule \\
Equation & \begin{displaymath} 
        q\left(t\right) = h(T(t) - T_{\text{env}}(t))
        \end{displaymath}
\\ \midrule \\
Description & \begin{symbDescription}
              \item{$q$ is the thermal flux normal to the surface ($\frac{\text{W}}{\text{m}^{2}}$)}
              \item{$t$ is the time (${\text{s}}$)}
              \item{$h$ is the convective heat transfer coefficient ($\frac{\text{W}}{\text{m}^{2}{}^{\circ}\text{C}}$)}
              \item{$T$ is the temperature of the body (${{}^{\circ}\text{C}}$)}
              \item{$T_{\text{env}}$ is the temperature of the environment surrounding the body (${{}^{\circ}\text{C}}$)}
              \end{symbDescription}
\\ \midrule \\

\\ \midrule \\
Source & --
         
\\ \midrule \\
RefBy & \wss{?}
        
\\ \bottomrule
\end{tabular}
\end{minipage}

%%%%%%%%%%%%%%%%%%%%%%%%%%%%%%%%%%%%%%%%%%%%%%%%%%%%%%%%%%%%%%%%%%%%%%%%%%%%%%%%%%

\paragraph{Preconditions for \hyperref[RT:nwtnCooling:ReduceDep]{RT:nwtnCooling:ReduceDep}:}
\label{RT:nwtnCooling:ReduceDepPrecond}

\begin{itemize}
\item \hyperref[assumpUnifHeatTransCoeff]{A:UniformHeatTransCoeffOverSurf},
\item \hyperref[assumpUnifTempSurf]{A:UniformTempOverSurf},
\item \hyperref[assumpUnifEnviroTempSurf]{A:UniformEnviroTemp},
\item \hyperref[assumpHeatTransIndepTime]{A:HeatTransCoeffIndepTime},
\item \hyperref[assumpHeatTransIndepTemp]{A:HeatTransCoeffIndepTemp}
\end{itemize}

%%%%%%%%%%%%%%%%%%%%%%%%%%%%%%%%%%%%%%

\paragraph{Detailed derivation of simplified Newton's Law of Cooling:}
\label{RT:nwtnCooling:ReduceDepDeriv}

Starting from \hyperref[BT:nwtnCooling]{BT:nwtnCooling}:

\begin{displaymath} 
        \textbf{q}\left(\textbf{s}, t\right) \cdot
        \symbf{\hat{n}} = h(\textbf{s}, t, T) (T(\textbf{s}, t) -
        T_{\text{env}}(\textbf{s}, t))
\end{displaymath}

\noindent First introduce $q$ as the normal flux to the surface $q(\textbf{s},
t) = \textbf{q}\left(\textbf{s}, t\right) \cdot \symbf{\hat{n}}$ and substitute
this into the above equation for:

\begin{displaymath} 
        q(\textbf{s}, t) = h(\textbf{s}, t, T) (T(\textbf{s}, t) -
        T_{\text{env}}(\textbf{s}, t))
\end{displaymath}

\noindent Apply the assumption that the heat transfer coefficient is uniform
over the surface (\hyperref[assumpUnifHeatTransCoeff]{A:UniformHeatTransCoeffOverSurf}):

\begin{displaymath} 
        q(\textbf{s}, t) = h(t, T) (T(\textbf{s}, t) -
        T_{\text{env}}(\textbf{s}, t))
\end{displaymath}

\noindent Apply the assumption that the temperature is uniform
over the surface (\hyperref[assumpUnifTempSurf]{A:UniformTempOverSurf}):

\begin{displaymath} 
        q(\textbf{s}, t) = h(t, T) (T(t) - T_{\text{env}}(\textbf{s}, t))
\end{displaymath}

\noindent Apply the assumption that the temperature of the environment is uniform
over the surface (\hyperref[assumpUnifEnviroTempSurf]{A:UniformEnviroTemp}):

\begin{displaymath} 
        q(\textbf{s}, t) = h(t, T) (T(t) - T_{\text{env}}(t))
\end{displaymath}

\noindent Since the right hand side of the equation does not depend on the
location on the surface ($\textbf{s}$), this dependence can be removed from the
left hand side of the equation.:

\begin{displaymath} 
        q(t) = h(t, T) (T(t) - T_{\text{env}}(t))
\end{displaymath}

\noindent Apply the assumption that the heat transfer coefficient is independent of time (\hyperref[assumpHeatTransIndepTime]{A:HeatTransCoeffIndepTime}):

\begin{displaymath} 
        q(t) = h(T) (T(t) - T_{\text{env}}(t))
\end{displaymath}

\noindent Apply the assumption that the heat transfer coefficient is independent of temperature (\hyperref[assumpHeatTransIndepTemp]{A:HeatTransCoeffIndepTemp}):

\begin{displaymath} 
        q(t) = h (T(t) - T_{\text{env}}(t))
\end{displaymath}

%%%%%%%%%%%%%%%%%%%%%%%%%%%%%%%%%%%%%%%%%%%%%%%%%%%%%%%%%%%%%%%%%%%%%%%%%%%%%%%%%%

\vspace{\baselineskip}
\noindent
\begin{minipage}{\textwidth}
\begin{tabular}{>{\raggedright}p{0.13\textwidth}>{\raggedright\arraybackslash}p{0.82\textwidth}}
\toprule \textbf{Refname} & \textbf{RT:htFluxWaterFromCoil}
\phantomsection 
\label{RT:htFluxWaterFromCoil}
\\ \midrule \\
Label & Heat flux into the water from the coil
        
\\ \midrule \\
Units & $\frac{\text{W}}{\text{m}^{2}}$
        
\\ \midrule \\
Equation & \begin{displaymath}
           {q_{\text{C}}(t)}={h_{\text{C}}} \left({T_{\text{C}}}-{T_{\text{W}}}\left(t\right)\right)
           \end{displaymath}
\\ \midrule \\
Description & \begin{symbDescription}
              \item{${q_{\text{C}}}$ is the heat flux into the water from the coil ($\frac{\text{W}}{\text{m}^{2}}$)}
              \item{${h_{\text{C}}}$ is the convective heat transfer coefficient between coil and water ($\frac{\text{W}}{\text{m}^{2}{}^{\circ}\text{C}}$)}
              \item{${T_{\text{C}}}$ is the temperature of the heating coil (${{}^{\circ}\text{C}}$)}
              \item{${T_{\text{W}}}$ is the temperature of the water (${{}^{\circ}\text{C}}$)}
              \item{$t$ is the time (${\text{s}}$)}
              \end{symbDescription}
\\ \midrule \\
Notes & ${q_{\text{C}}}$ is found by assuming that Newton's law of cooling
applies (\hyperref[assumpLCCCW]{A:Newton-Law-Convective-Cooling-Coil-Water}).
        
\\ \midrule \\
Source & \cite{koothoor2013}
                  
\\ \midrule \\
RefBy & \hyperref[FT:eBalanceOnWtr]{FT:eBalanceOnWtr}
        
\\ \bottomrule
\end{tabular}
\end{minipage}

%%%%%%%%%%%%%%%%%%%%%%%%%%%%%%%%%%%%%%%%%%%%%%%%%%%%%%%%%%%%%%%%%%%%%%%%%%%%%%%%%%

\paragraph{Preconditions for \hyperref[RT:htFluxWaterFromCoil]{RT:htFluxWaterFromCoil}:}
\label{RT:htFluxWaterFromCoilPrecond}

\begin{itemize}
\item \hyperref[assumpUnifHeatTransCoeffCoil]{A:UniformHeatTransCoeffOverCoil}, 
\item \hyperref[assumpTHCCoL]{A:Temp-Heating-Coil-Constant-over-Length}, 
\item \hyperref[assumpFullyMixed]{A:UniformFullyMixed}, 
\item \hyperref[assumpHeatTransCoilIndepTime]{A:HeatTransCoeffCoilIndepTime}, 
\item \hyperref[assumpHeatTransCoilIndepTemp]{A:HeatTransCoeffCoilIndepTemp}, 
\item \hyperref[assumpTHCCoT]{A:Temp-Heating-Coil-Constant-over-Time}
\end{itemize}

%%%%%%%%%%%%%%%%%%%%%%%%%%%%%%%%%%%%%%

\paragraph{Detailed derivation of heat flux into the water from the coil:}
\label{RT:htFluxWaterFromCoilDeriv}

We assume that Newton's law of cooling applies between the coil and the water
(\hyperref[assumpLCCCW]{A:Newton-Law-Convective-Cooling-Coil-Water}).  To use
the simplified form of Newton's law of cooling
(\hyperref[RT:nwtnCooling:ReduceDep]{RT:nwtnCooling:ReduceDep}) we need to map
the coil and water specific variables to the general form of the equation and we
need to show that the pre-conditions are satisfied.

The beginning equation is from
\hyperref[RT:nwtnCooling:ReduceDep]{RT:nwtnCooling:ReduceDep}:

\begin{displaymath} 
        q(t) = h (T(t) - T_{\text{env}}(t))
\end{displaymath}

We apply this equation with the surface as the surface of the coil in the tank
of water.  The environment is the tank of water and the body under consideration
is the heating coil.  Therefore, to apply Newton's law of cooling in the current
context, we use the following mapping: $q(t) = q_\text{C}(t)$, $h =
h_{\text{C}}$, $T(t) = T_{\text{C}}$ and $T_\text{env}(t) = T_\text{W}(t)$.
With these substitions we have the following equation:

\begin{displaymath}
        {q_{\text{C}}}(t)={h_{\text{C}}} \left({T_{\text{C}}(t)}-{T_{\text{W}}}\left(t\right)\right)
\end{displaymath}

To use this equation, we need to show that the pre-conditions are satisfied by
our system.  We will cover each assumption from \hyperref[RT:nwtnCooling:ReduceDep]{RT:nwtnCooling:ReduceDep} in turn.

\begin{itemize}
\item We can use
\hyperref[assumpUnifHeatTransCoeff]{A:UniformHeatTransCoeffOverSurf} because we
make the assumption that the heat transfer coefficient over the surface of the
coil is uniform
(\hyperref[assumpUnifHeatTransCoeffCoil]{A:UniformHeatTransCoeffOverCoil}).
\item We can apply \hyperref[assumpUnifTempSurf]{A:UniformTempOverSurf} because
of our assumption that the temperature of the coil is the same throughout its
length (\hyperref[assumpTHCCoL]{A:Temp-Heating-Coil-Constant-over-Length}).
\item We can apply the assumption that the environment temperature of the tank
is uniform over the surface
(\hyperref[assumpUnifEnviroTempSurf]{A:UniformEnviroTemp}) because we are
assuming that the tank is fully mixed
(\hyperref[assumpFullyMixed]{A:UniformFullyMixed}).
\item We can apply
\hyperref[assumpHeatTransIndepTime]{A:HeatTransCoeffIndepTime} because we assume
that the heat transfer coefficient for the coil does not change over time
(\hyperref[assumpHeatTransCoilIndepTime]{A:HeatTransCoeffCoilIndepTime}).
\item We can apply
\hyperref[assumpHeatTransIndepTemp]{A:HeatTransCoeffIndepTemp} because we assume
for the coil that the heat transfer coefficient does not vary with the
temperature
(\hyperref[assumpHeatTransCoilIndepTemp]{A:HeatTransCoeffCoilIndepTemp}).
\end{itemize}

In addition to the above assumptions we make one more simplifying assumption. We
assume that the temperature in the heating coil is constant over time
(\hyperref[assumpTHCCoT]{A:Temp-Heating-Coil-Constant-over-Time}).  Our
rationale is that the simulations are for one heating event where the conditions
will not overly vary.  With $T_\text{C}$ now independent of temperature, the
equation can be rewritten as:

\begin{displaymath}
        {q_{\text{C}}}(t)={h_{\text{C}}} \left({T_{\text{C}}}-{T_{\text{W}}}\left(t\right)\right)
\end{displaymath}

%%%%%%%%%%%%%%%%%%%%%%%%%%%%%%%%%%%%%%%%%%%%%%%%%%%%%%%%%%%%%%%%%%%%%%%%%%%%%%%%%%

\section{Definitions}
\label{Sec:DDs}
This section collects and defines all the data needed to build the instance models.

\vspace{\baselineskip}
\noindent
\begin{minipage}{\textwidth}
\begin{tabular}{>{\raggedright}p{0.13\textwidth}>{\raggedright\arraybackslash}p{0.82\textwidth}}
\toprule \textbf{Refname} & \textbf{DD:waterMass}
\phantomsection 
\label{DD:waterMass}
\\ \midrule \\
Label & Mass of water
        
\\ \midrule \\
Symbol & ${m_{\text{W}}}$
         
\\ \midrule \\
Units & ${\text{kg}}$
        
\\ \midrule \\
Equation & \begin{displaymath}
           {m_{\text{W}}}={V_{\text{W}}} {ρ_{\text{W}}}
           \end{displaymath}
\\ \midrule \\
Description & \begin{symbDescription}
              \item{${m_{\text{W}}}$ is the mass of water (${\text{kg}}$)}
              \item{${V_{\text{W}}}$ is the volume of water (${\text{m}^{3}}$)}
              \item{${ρ_{\text{W}}}$ is the density of water ($\frac{\text{kg}}{\text{m}^{3}}$)}
              \end{symbDescription}
\\ \midrule \\
Notes & Density ($ρ$) is defined in \hyperref[BT:density]{BT:density}
\\ \midrule \\
Source & --
         
\\ \midrule \\
RefBy & \hyperref[findMass]{FR:Find-Mass}
        
\\ \bottomrule
\end{tabular}
\end{minipage}

%%%%%%%%%%%%%%%%%%%%%%%%%%%%%%%%%%%%%%

\vspace{\baselineskip}
\noindent
\begin{minipage}{\textwidth}
\begin{tabular}{>{\raggedright}p{0.13\textwidth}>{\raggedright\arraybackslash}p{0.82\textwidth}}
\toprule \textbf{Refname} & \textbf{DD:waterVolume.nopcm}
\phantomsection 
\label{DD:waterVolume.nopcm}
\\ \midrule \\
Label & Volume of water
        
\\ \midrule \\
Symbol & ${V_{\text{W}}}$
         
\\ \midrule \\
Units & ${\text{m}^{3}}$
        
\\ \midrule \\
Equation & \begin{displaymath}
           {V_{\text{W}}}={V_{\text{tank}}}
           \end{displaymath}
\\ \midrule \\
Description & \begin{symbDescription}
              \item{${V_{\text{W}}}$ is the volume of water (${\text{m}^{3}}$)}
              \item{${V_{\text{tank}}}$ is the volume of the cylindrical tank (${\text{m}^{3}}$)}
              \end{symbDescription}
\\ \midrule \\
Notes & Based on \hyperref[assumpVCN]{A:Volume-Coil-Negligible}. ${V_{\text{tank}}}$ is defined in \hyperref[DD:tankVolume]{DD:tankVolume}.
        
\\ \midrule \\
Source & --
         
\\ \midrule \\
RefBy & \hyperref[findMass]{FR:Find-Mass}
        
\\ \bottomrule
\end{tabular}
\end{minipage}

%%%%%%%%%%%%%%%%%%%%%%%%%%%%%%%%%%%%%%

\vspace{\baselineskip}
\noindent
\begin{minipage}{\textwidth}
\begin{tabular}{>{\raggedright}p{0.13\textwidth}>{\raggedright\arraybackslash}p{0.82\textwidth}}
\toprule \textbf{Refname} & \textbf{DD:tankVolume}
\phantomsection 
\label{DD:tankVolume}
\\ \midrule \\
Label & Volume of the cylindrical tank
        
\\ \midrule \\
Symbol & ${V_{\text{tank}}}$
         
\\ \midrule \\
Units & ${\text{m}^{3}}$
        
\\ \midrule \\
Equation & \begin{displaymath}
           {V_{\text{tank}}}=π \left(\frac{D}{2}\right)^{2} L
           \end{displaymath}
\\ \midrule \\
Description & \begin{symbDescription}
              \item{${V_{\text{tank}}}$ is the volume of the cylindrical tank (${\text{m}^{3}}$)}
              \item{$π$ is the ratio of circumference to diameter for any circle (Unitless)}
              \item{$D$ is the diameter of tank (${\text{m}}$)}
              \item{$L$ is the length of tank (${\text{m}}$)}
              \end{symbDescription}
\\ \midrule \\
Source & --
         
\\ \midrule \\
RefBy & \hyperref[DD:waterVolume.nopcm]{DD:waterVolume\_nopcm} and \hyperref[findMass]{FR:Find-Mass}
        
\\ \bottomrule
\end{tabular}
\end{minipage}

%%%%%%%%%%%%%%%%%%%%%%%%%%%%%%%%%%%%%%

\vspace{\baselineskip}
\noindent
\begin{minipage}{\textwidth}
\begin{tabular}{>{\raggedright}p{0.13\textwidth}>{\raggedright\arraybackslash}p{0.82\textwidth}}
\toprule \textbf{Refname} & \textbf{DD:balanceDecayRate}
\phantomsection 
\label{DD:balanceDecayRate}
\\ \midrule \\
Label & ODE parameter for water related to decay time
        
\\ \midrule \\
Symbol & ${τ_{\text{W}}}$
         
\\ \midrule \\
Units & ${\text{s}}$
        
\\ \midrule \\
Equation & \begin{displaymath}
           {τ_{\text{W}}}=\frac{{m_{\text{W}}} {C_{\text{W}}}}{{h_{\text{C}}} {A_{\text{C}}}}
           \end{displaymath}
\\ \midrule \\
Description & \begin{symbDescription}
              \item{${τ_{\text{W}}}$ is the ODE parameter for water related to decay time (${\text{s}}$)}
              \item{${m_{\text{W}}}$ is the mass of water (${\text{kg}}$)}
              \item{${C_{\text{W}}}$ is the specific heat capacity of water ($\frac{\text{J}}{\text{kg}{}^{\circ}\text{C}}$)}
              \item{${h_{\text{C}}}$ is the convective heat transfer coefficient between coil and water ($\frac{\text{W}}{\text{m}^{2}{}^{\circ}\text{C}}$)}
              \item{${A_{\text{C}}}$ is the heating coil surface area (${\text{m}^{2}}$)}
              \end{symbDescription}
\\ \midrule \\
Source & \cite{koothoor2013}
         
\\ \midrule \\
RefBy & \hyperref[outputInputDerivVals]{FR:Output-Input-Derived-Values} and \hyperref[FT:eBalanceOnWtr]{FT:eBalanceOnWtr}
        
\\ \bottomrule
\end{tabular}
\end{minipage}

%%%%%%%%%%%%%%%%%%%%%%%%%%%%%%%%%%%%%%%%%%%%%%%%%%%%%%%%%%%%%%%%%%%%%%%%%%%%%%%%%%

\section{Final Theories}
\label{Sec:FTs}
This section transforms the problem defined in the
\hyperref[Sec:ProbDesc]{problem description} into one which is expressed in
mathematical terms. It uses concrete symbols defined in the
\hyperref[Sec:DDs]{data definitions} to replace the abstract symbols in the
models identified in \hyperref[Sec:BTs]{theoretical models} and
\hyperref[Sec:RefndTheories]{general definitions}.

The goal \hyperref[waterTempGS]{GS:Predict-Water-Temperature} is met by
\hyperref[FT:eBalanceOnWtr]{FT:eBalanceOnWtr} and the goal
\hyperref[waterEnergyGS]{GS:Predict-Water-Energy} is met by
\hyperref[FT:heatEInWtr]{FT:heatEInWtr}.

%%%%%%%%%%%%%%%%%%%%%%%%%%%%%%%%%%%%%%

\vspace{\baselineskip}
\noindent
\begin{minipage}{\textwidth}
\begin{tabular}{>{\raggedright}p{0.13\textwidth}>{\raggedright\arraybackslash}p{0.82\textwidth}}
\toprule \textbf{Refname} & \textbf{FT:eBalanceOnWtr}
\phantomsection 
\label{FT:eBalanceOnWtr}
\\ \midrule \\
Label & Energy balance on water to find the temperature of the water
        
\\ \midrule \\
Input & ${T_{\text{C}}}$, ${T_{\text{init}}}$, ${t_{\text{final}}}$, ${A_{\text{C}}}$, ${h_{\text{C}}}$, ${C_{\text{W}}}$, ${m_{\text{W}}}$
        
\\ \midrule \\
Output & ${T_{\text{W}}}$
         
\\ \midrule \\
Input Constraints & \begin{displaymath}
                    {T_{\text{C}}}\geq{}{T_{\text{init}}}
                    \end{displaymath}
\\ \midrule \\
Output Constraints & 
\\ \midrule \\
Equation & \begin{displaymath}
           \frac{\,d{T_{\text{W}}}(t)}{\,dt}=\frac{1}{{τ_{\text{W}}}} \left({T_{\text{C}}}-{T_{\text{W}}}\left(t\right)\right)
           \end{displaymath}
\\ \midrule \\
Description & \begin{symbDescription}
              \item{$t$ is the time (${\text{s}}$)}
              \item{${T_{\text{W}}}$ is the temperature of the water (${{}^{\circ}\text{C}}$)}
              \item{${τ_{\text{W}}}$ is the ODE parameter for water related to decay time (${\text{s}}$)}
              \item{${T_{\text{C}}}$ is the temperature of the heating coil (${{}^{\circ}\text{C}}$)}
              \end{symbDescription}
\\ \midrule \\
Notes & \wss{The initial temperature of the water in the tank is
$T_\text{init}$;  that is, $T_\text{W}(0) = T_\text{init}$.}  ${τ_{\text{W}}}$
is calculated from \hyperref[DD:balanceDecayRate]{DD:balanceDecayRate}.
        
The above equation applies as long as the water is in liquid form,
$0\lt{}{T_{\text{W}}}\lt{}100$ (${{}^{\circ}\text{C}}$) where $0$
(${{}^{\circ}\text{C}}$) and $100$ (${{}^{\circ}\text{C}}$) are the melting and
boiling point temperatures of water, respectively
(\hyperref[assumpWAL]{A:Water-Always-Liquid}).
        
\\ \midrule \\
Source & \cite[(with PCM removed)]{koothoor2013}

\\ \midrule \\
RefBy & \hyperref[unlikeChgNIHG]{UC:No-Internal-Heat-Generation},
\hyperref[findMass]{FR:Find-Mass}, and
\hyperref[calcTempWtrOverTime]{FR:Calculate-Temperature-Water-Over-Time}
        
\\ \bottomrule
\end{tabular}
\end{minipage}

%%%%%%%%%%%%%%%%%%%%%%%%%%%%%%%%%%%%%%

\paragraph{Preconditions for \hyperref[FT:eBalanceOnWtr]{FT:eBalanceOnWtr}:}
\label{FT:eBalanceOnWtrPrecond}

\begin{itemize}
\item \hyperref[assumpTEO]{A:Thermal-Energy-Only},
\item \hyperref[assumpFullyMixed]{A:UniformFullyMixed}
\item \hyperref[assumpNIHGBW]{A:No-Internal-Heat-Generation-By-Water},
\item \hyperref[assumpPIT]{A:Perfect-Insulation-Tank},
\item \hyperref[assumpDensIndepTWater]{A:DensityIndepTempWater},
\item \hyperref[assumpSpecHeatIndepTWater]{A:SpecHeatIndepTempWater},
\item \hyperref[assumpUnifHeatTransCoeffCoil]{A:UniformHeatTransCoeffOverCoil},
\item \hyperref[assumpTHCCoL]{A:Temp-Heating-Coil-Constant-over-Length},
\item \hyperref[assumpHeatTransCoilIndepTime]{A:HeatTransCoeffCoilIndepTime},
\item \hyperref[assumpHeatTransCoilIndepTemp]{A:HeatTransCoeffCoilIndepTemp}
\end{itemize}

%%%%%%%%%%%%%%%%%%%%%%%%%%%%%%%%%%%%%%

\paragraph{Detailed derivation of the energy balance on water:}
\label{FT:eBalanceOnWtrDeriv}

To derive the ODE that defines the temperature of the water over time, we start
with the simplified rate of change of temperature equation
(\hyperref[RT:rocTempSimp]{RT:rocTempSimp}):

\begin{displaymath}
m C \frac{\,dT(t)}{\,dt} = -\sum_{i=0}^n {q_i(t)} {A_i} + g(t) V
\end{displaymath}

The volume under consideration is the water tank.  There are two surfaces
($n=2$) that encompass the volume: the walls of the tank and the surface of the
heating coil.  We will represent the thermal flux through the tank walls by
$q_\text{tank}$ and the thermal flux through the heating coil by $q_\text{C}$.
The respective surface areas are $A_\text{tank}$ and $A_\text{C}$. To complete
the application of \hyperref[RT:rocTempSimp]{RT:rocTempSimp} to the water tank
and coil system, we add the following symbols: 

\begin{itemize}
\item $m_\text{W}$ for the mass of the water in the tank, $m = m_\text{W}$.
\item $C_\text{W}$ for the specific heat capacity of the water, $C =
C_\text{W}$.
\item $T_\text{W}$ for the temperature of the water in the tank, $T =
m_\text{W}$.
\item $g_\text{W}$ for the per unit heat generation of the water in the tank, $g
= g_\text{W}$.
\end{itemize}

Substituting the above variables into \hyperref[RT:rocTempSimp]{RT:rocTempSimp}, we have:

\begin{displaymath}
m_\text{W} C_\text{W} \frac{\,dT_\text{W}(t)}{\,dt} = -{q_\text{tank}(t)} {A_\text{tank}} -{q_\text{C}(t)} {A_\text{C}} + g_\text{W}(t) V
\end{displaymath}

Apply the assumption that there is no internal heat generation for the water
($g_\text{W} = 0$) (\hyperref[assumpNIHGBW]{A:No-Internal-Heat-Generation-By-Water}):

\begin{displaymath}
m_\text{W} C_\text{W} \frac{\,dT_\text{W}(t)}{\,dt} = -{q_\text{tank}(t)} {A_\text{tank}} -{q_\text{C}(t)} {A_\text{C}}
\end{displaymath}

Apply the assumption that the tank is a perfect insulator ($q_\text{tank}=0$)
(\hyperref[assumpPIT]{A:Perfect-Insulation-Tank}): 

\begin{displaymath}
m_\text{W} C_\text{W} \frac{\,dT_\text{W}(t)}{\,dt} = -{q_\text{C}(t)} {A_\text{C}}
\end{displaymath}
        
We find the value for ${q_{\text{C}}}$ from
(\hyperref[RT:htFluxWaterFromCoil]{RT:htFluxWaterFromCoil}):
${q_{\text{C}}(t)}={h_{\text{C}}}
\left({T_{\text{C}}}-{T_{\text{W}}}\left(t\right)\right)$.  However before
substitution into the energy balance equation, we need to change the sign of
${q_{\text{C}}}$ because it is in the opposite direction to the unit outward
normal for the surface.  The thermal flux is from the coil to the water because
of the assumption that the coil is warmer than the water tank ($T_\text{C} \geq
T_\text{init}$).  Therefore, the equation can be rewritten as:

\begin{displaymath}
m_\text{W} C_\text{W} \frac{\,dT_\text{W}(t)}{\,dt} = {h_{\text{C}}}
\left({T_{\text{C}}}-{T_{\text{W}}}\left(t\right)\right) {A_\text{C}}
\end{displaymath}
        
Move $A_\text{C}$:

\begin{displaymath}
m_\text{W} C_\text{W} \frac{\,dT_\text{W}(t)}{\,dt} = {h_{\text{C}}} {A_\text{C}}
\left({T_{\text{C}}}-{T_{\text{W}}}\left(t\right)\right) 
\end{displaymath}
        
Dividing both sides of the above equation by ${m_{\text{W}}} {C_{\text{W}}}$, we
obtain:

\begin{displaymath}
\frac{\,d{T_{\text{W}}}(t)}{\,dt}=\frac{{h_{\text{C}}} {A_{\text{C}}}}{{m_{\text{W}}} {C_{\text{W}}}} \left({T_{\text{C}}}-{T_{\text{W}}(t)}\right)
\end{displaymath}

By substituting ${τ_{\text{W}}}$ (from \hyperref[DD:balanceDecayRate]{DD:balanceDecayRate}), this can be rewritten as:

\begin{displaymath}
\frac{\,d{T_{\text{W}}}(t)}{\,dt}=\frac{1}{{τ_{\text{W}}}} \left({T_{\text{C}}}-{T_{\text{W}}(t)}\right)
\end{displaymath}

To use this equation, we need to show that the pre-conditions from
\hyperref[RT:rocTempSimp]{RT:rocTempSimp} and
\hyperref[RT:htFluxWaterFromCoil]{RT:htFluxWaterFromCoil} are satisfied.

\begin{itemize}
\item \hyperref[assumpTEO]{A:Thermal-Energy-Only}: We will satisfy this
assumption by making the same assumption.
\item \hyperref[assumpUnifDens]{A:UniformDensity-OverVol}: we make this
assumption because we assume that the tank is fully mixed
(\hyperref[assumpFullyMixed]{A:UniformFullyMixed}).
\item \hyperref[assumpUnifSpecHeat]{A:UniformSpecHeat-OverVol}: we make this
assumption because we assume that the tank is fully mixed
(\hyperref[assumpFullyMixed]{A:UniformFullyMixed}).
\item \hyperref[assumpUnifTemp]{A:UniformTemp-OverVol}: we make this assumption
because we assume that the tank is fully mixed
(\hyperref[assumpFullyMixed]{A:UniformFullyMixed}).
\item \hyperref[assumpUnifHeatGen]{A:UniformHeatGen-OverVol}: we make this
assumption because we assume that heat generation for the water is zero over the
volume (\hyperref[assumpNIHGBW]{A:No-Internal-Heat-Generation-By-Water}).
\item \hyperref[assumpDensIndepT]{A:DensityIndepTemp}: we assume that for the water in the tank, and the temperature bounds of the simulation, the density does not change with temperature (\hyperref[assumpDensIndepTWater]{A:DensityIndepTempWater}).  \wss{We could cite actually data for this.}
\item \hyperref[assumpSpecHeatIndepT]{A:SpecHeatIndepTemp}: we assume that for
the water in the tank, and the temperature bounds of the simulation, the
specific heat capacity does not change with temperature
(\hyperref[assumpSpecHeatIndepTWater]{A:SpecHeatIndepTempWater}).  \wss{We could
cite actually data for this.}
\item \hyperref[assumpVolCompact]{A:VolumeIsCompact}: The shape of the tank
is compact, so this assumption is satisfied.  \wss{Do we need more here?}
\item \hyperref[assumpPiecewiseSmooth]{A:VolHasPiecewise-Smooth-Surface}: The
shape of the surface of the tank is smooth, as is the surface of the heating
coil.  \wss{Do we need more here?}
\item \hyperref[assumpUnifThermFlux]{A:ThermalFlux-UniformOverSurfaces}: This
assumption is satisfied because of the following assumptions that are
pre-conditions for \hyperref[RT:htFluxWaterFromCoil]{RT:htFluxWaterFromCoil}:
\hyperref[assumpUnifHeatTransCoeffCoil]{A:UniformHeatTransCoeffOverCoil},
\hyperref[assumpTHCCoL]{A:Temp-Heating-Coil-Constant-over-Length},
\hyperref[assumpFullyMixed]{A:UniformFullyMixed}.
\item \hyperref[assumpUnifHeatTransCoeffCoil]{A:UniformHeatTransCoeffOverCoil}:
We will satisfy this assumption by making the same assumption.
\item \hyperref[assumpTHCCoL]{A:Temp-Heating-Coil-Constant-over-Length}: We will
satisfy this assumption by making the same assumption.
\item \hyperref[assumpFullyMixed]{A:UniformFullyMixed}: We will satisfy this
assumption by making the same assumption.
\item \hyperref[assumpHeatTransCoilIndepTime]{A:HeatTransCoeffCoilIndepTime}: We
will satisfy this assumption by making the same assumption.
\item \hyperref[assumpHeatTransCoilIndepTemp]{A:HeatTransCoeffCoilIndepTemp}: We
will satisfy this assumption by making the same assumption.
\item \hyperref[assumpTHCCoT]{A:Temp-Heating-Coil-Constant-over-Time}: We will
satisfy this assumption by making the same assumption.
\end{itemize}

%%%%%%%%%%%%%%%%%%%%%%%%%%%%%%%%%%%%%%

\vspace{\baselineskip}
\noindent
\begin{minipage}{\textwidth}
\begin{tabular}{>{\raggedright}p{0.13\textwidth}>{\raggedright\arraybackslash}p{0.82\textwidth}}
\toprule \textbf{Refname} & \textbf{FT:heatEInWtr}
\phantomsection 
\label{FT:heatEInWtr}
\\ \midrule \\
Label & Heat energy in the water
        
\\ \midrule \\
Input & ${T_{\text{init}}}$, ${m_{\text{W}}}$, ${C_{\text{W}}}$, ${m_{\text{W}}}$
        
\\ \midrule \\
Output & ${E_{\text{W}}}$
         
\\ \midrule \\
Input Constraints & 
\\ \midrule \\
Output Constraints & 
\\ \midrule \\
Equation & \begin{displaymath}
           {E_{\text{W}}}\left(t\right)={C_{\text{W}}} {m_{\text{W}}} \left({T_{\text{W}}}\left(t\right)-{T_{\text{init}}}\right)
           \end{displaymath}
\\ \midrule \\
Description & \begin{symbDescription}
              \item{${E_{\text{W}}}$ is the change in heat energy in the water (${\text{J}}$)}
              \item{$t$ is the time (${\text{s}}$)}
              \item{${C_{\text{W}}}$ is the specific heat capacity of water ($\frac{\text{J}}{\text{kg}{}^{\circ}\text{C}}$)}
              \item{${m_{\text{W}}}$ is the mass of water (${\text{kg}}$)}
              \item{${T_{\text{W}}}$ is the temperature of the water (${{}^{\circ}\text{C}}$)}
              \item{${T_{\text{init}}}$ is the initial temperature (${{}^{\circ}\text{C}}$)}
              \end{symbDescription}
\\ \midrule \\
Notes & The above equation is derived using \hyperref[BT:sensHtE]{BT:sensHtE}.
        
        The change in temperature is the difference between the temperature at time $t$ (${\text{s}}$), ${T_{\text{W}}}$ and the initial temperature, ${T_{\text{init}}}$ (${{}^{\circ}\text{C}}$).
        
        This equation applies as long as $0\lt{}{T_{\text{W}}}\lt{}100$${{}^{\circ}\text{C}}$ (\hyperref[assumpWAL]{A:Water-Always-Liquid}, \hyperref[assumpAPT]{A:Atmospheric-Pressure-Tank}).
        
\\ \midrule \\
Source & \cite{koothoor2013}
         
\\ \midrule \\
RefBy & \hyperref[calcChgHeatEnergyWtrOverTime]{FR:Calculate-Change-Heat\_Energy-Water-Over-Time}
        
\\ \bottomrule
\end{tabular}
\end{minipage}

\subsubsection{Data Constraints}
\label{Sec:DataConstraints}
The \hyperref[Table:InDataConstraints]{Data Constraints Table} shows the data constraints on the input variables. The column for physical constraints gives the physical limitations on the range of values that can be taken by the variable. The uncertainty column provides an estimate of the confidence with which the physical quantities can be measured. This information would be part of the input if one were performing an uncertainty quantification exercise. The constraints are conservative, to give the user of the model the flexibility to experiment with unusual situations. The column of typical values is intended to provide a feel for a common scenario. The column for software constraints restricts the range of inputs to reasonable values.

\begin{longtable}{l l l l l}
\toprule
\textbf{Var} & \textbf{Physical Constraints} & \textbf{Software Constraints} & \textbf{Typical Value} & \textbf{Uncert.}
\\
\midrule
\endhead
${A_{\text{C}}}$ & ${A_{\text{C}}}\gt{}0$ & ${A_{\text{C}}}\leq{}{{A_{\text{C}}}^{\text{max}}}$ & $0.12$ ${\text{m}^{2}}$ & 10$\%$
\\
${C_{\text{W}}}$ & ${C_{\text{W}}}\gt{}0$ & ${{C_{\text{W}}}^{\text{min}}}\lt{}{C_{\text{W}}}\lt{}{{C_{\text{W}}}^{\text{max}}}$ & $4186$ $\frac{\text{J}}{\text{kg}{}^{\circ}\text{C}}$ & 10$\%$
\\
$D$ & $D\gt{}0$ & ${AR_{\text{min}}}\leq{}D\leq{}{AR_{\text{max}}}$ & $0.412$ ${\text{m}}$ & 10$\%$
\\
${h_{\text{C}}}$ & ${h_{\text{C}}}\gt{}0$ & ${{h_{\text{C}}}^{\text{min}}}\leq{}{h_{\text{C}}}\leq{}{{h_{\text{C}}}^{\text{max}}}$ & $1000$ $\frac{\text{W}}{\text{m}^{2}{}^{\circ}\text{C}}$ & 10$\%$
\\
$L$ & $L\gt{}0$ & ${L_{\text{min}}}\leq{}L\leq{}{L_{\text{max}}}$ & $1.5$ ${\text{m}}$ & 10$\%$
\\
${T_{\text{C}}}$ & $0\lt{}{T_{\text{C}}}\lt{}100$ & -- & $50$ ${{}^{\circ}\text{C}}$ & 10$\%$
\\
${T_{\text{init}}}$ & $0\lt{}{T_{\text{init}}}\lt{}100$ & -- & $40$ ${{}^{\circ}\text{C}}$ & 10$\%$
\\
${t_{\text{final}}}$ & ${t_{\text{final}}}\gt{}0$ & ${t_{\text{final}}}\lt{}{{t_{\text{final}}}^{\text{max}}}$ & $50000$ ${\text{s}}$ & 10$\%$
\\
${t_{\text{step}}}$ & $0\lt{}{t_{\text{step}}}\lt{}{t_{\text{final}}}$ & -- & $0.01$ ${\text{s}}$ & 10$\%$
\\
${ρ_{\text{W}}}$ & ${ρ_{\text{W}}}\gt{}0$ & ${{ρ_{\text{W}}}^{\text{min}}}\lt{}{ρ_{\text{W}}}\leq{}{{ρ_{\text{W}}}^{\text{max}}}$ & $1000$ $\frac{\text{kg}}{\text{m}^{3}}$ & 10$\%$
\\
\bottomrule
\caption{Input Data Constraints}
\label{Table:InDataConstraints}
\end{longtable}
\subsubsection{Properties of a Correct Solution}
\label{Sec:CorSolProps}
The \hyperref[Table:OutDataConstraints]{Data Constraints Table} shows the data constraints on the output variables. The column for physical constraints gives the physical limitations on the range of values that can be taken by the variable.

\begin{longtable}{l l}
\toprule
\textbf{Var} & \textbf{Physical Constraints}
\\
\midrule
\endhead
${T_{\text{W}}}$ & ${T_{\text{init}}}\leq{}{T_{\text{W}}}\leq{}{T_{\text{C}}}$
\\
${E_{\text{W}}}$ & ${E_{\text{W}}}\geq{}0$
\\
\bottomrule
\caption{Output Data Constraints}
\label{Table:OutDataConstraints}
\end{longtable}

\section{Rationale} \label{Sec:Rationale}

\wss{We might want to bring in the goal of the simulation, likely related to the
purpose section.  Some purposes do not require high fidelity simulations.}

\begin{itemize}
\item \hyperref[assumpUnifHeatTransCoeffCoil]{A:UniformHeatTransCoeffOverCoil}:
The rationale for this assumption is that heating coils are constructed so that
the effective heat transfer coefficient will be uniform.

\item \hyperref[assumpTHCCoL]{A:Temp-Heating-Coil-Constant-over-Length}: The
rationale for this is that although the temperature of the coil will drop over
its length, the change will not be large enough to matter because new heated
water will continually be moved into the coil.

\item \hyperref[assumpFullyMixed]{A:UniformFullyMixed}: Our rationale for assuming
that the tank is fully mixed is that the temperature changes are relatively slow
giving convective and conductive heat transfer enough time to even out the
temperature in the tank.  \wss{Maybe the rationale should go with the
assumption, since assumptions may be reused in other contexts.}
\wss{Assumptions are sometimes made even though they are known to likely be
inaccurate, but the scientist wants to explore the model.}

\item \hyperref[assumpHeatTransCoilIndepTime]{A:HeatTransCoeffCoilIndepTime}:
Our rationale for this assumption is that the physical characteristics of the
heating coil are not going to change appreciably over the length of any given
simulation.

\item \hyperref[assumpHeatTransCoilIndepTemp]{A:HeatTransCoeffCoilIndepTemp}:
Our rationale for this is that within the operating range of the coil the change
in the heat transfer coefficient with temperature is relatively small.

\item \hyperref[assumpTEO]{A:Thermal-Energy-Only}

\item \hyperref[assumpFullyMixed]{A:UniformFullyMixed}

\item \hyperref[assumpNIHGBW]{A:No-Internal-Heat-Generation-By-Water}

\item \hyperref[assumpPIT]{A:Perfect-Insulation-Tank}

\item \hyperref[assumpDensIndepTWater]{A:DensityIndepTempWater}

\item \hyperref[assumpSpecHeatIndepTWater]{A:SpecHeatIndepTempWater}

\end{itemize}
        
\section{Requirements} \label{Sec:Requirements}

This section provides the functional requirements, the tasks and behaviours that
the software is expected to complete, and the non-functional requirements, the
qualities that the software is expected to exhibit.

\subsection{Functional Requirements} \label{Sec:FRs}

This section provides the functional requirements, the tasks and behaviours that
the software is expected to complete.

\begin{itemize}
\item[Input-Initial-Values:\phantomsection\label{inputInitVals}]{Input the following values described in the table for \hyperref[Table:ReqInputs]{Required Inputs}, which define the tank parameters, material properties, and initial conditions.}
\item[Find-Mass:\phantomsection\label{findMass}]{Use the inputs in \hyperref[inputInitVals]{FR:Input-Initial-Values} to find the mass needed for \hyperref[FT:eBalanceOnWtr]{FT:eBalanceOnWtr}, using \hyperref[DD:waterMass]{DD:waterMass}, \hyperref[DD:waterVolume.nopcm]{DD:waterVolume\_nopcm}, and \hyperref[DD:tankVolume]{DD:tankVolume}.}
\item[Check-Input-with-Physical\_Constraints:\phantomsection\label{checkWithPhysConsts}]{Verify that the inputs satisfy the required \hyperref[Sec:DataConstraints]{physical constraints}.}
\item[Output-Input-Derived-Values:\phantomsection\label{outputInputDerivVals}]{Output the input values and derived values in the following list: the values (from \hyperref[inputInitVals]{FR:Input-Initial-Values}), the mass (from \hyperref[findMass]{FR:Find-Mass}), and ${τ_{\text{W}}}$ (from \hyperref[DD:balanceDecayRate]{DD:balanceDecayRate}).}
\item[Calculate-Temperature-Water-Over-Time:\phantomsection\label{calcTempWtrOverTime}]{Calculate and output the temperature of the water (${T_{\text{W}}}$($t$)) over the simulation time (from \hyperref[FT:eBalanceOnWtr]{FT:eBalanceOnWtr}).}
\item[Calculate-Change-Heat\_Energy-Water-Over-Time:\phantomsection\label{calcChgHeatEnergyWtrOverTime}]{Calculate and output the change in heat energy in the water (${E_{\text{W}}}$($t$)) over the simulation time (from \hyperref[FT:heatEInWtr]{FT:heatEInWtr}).}
\end{itemize}
\begin{longtabu}{l X[l] l}
\toprule
\textbf{Symbol} & \textbf{Description} & \textbf{Units}
\\
\midrule
\endhead
${A_{\text{C}}}$ & Heating coil surface area & ${\text{m}^{2}}$
\\
${A_{\text{tol}}}$ & Absolute tolerance & --
\\
${C_{\text{W}}}$ & Specific heat capacity of water & $\frac{\text{J}}{\text{kg}{}^{\circ}\text{C}}$
\\
$D$ & Diameter of tank & ${\text{m}}$
\\
${h_{\text{C}}}$ & Convective heat transfer coefficient between coil and water & $\frac{\text{W}}{\text{m}^{2}{}^{\circ}\text{C}}$
\\
$L$ & Length of tank & ${\text{m}}$
\\
${R_{\text{tol}}}$ & Relative tolerance & --
\\
${T_{\text{C}}}$ & Temperature of the heating coil & ${{}^{\circ}\text{C}}$
\\
${T_{\text{init}}}$ & Initial temperature & ${{}^{\circ}\text{C}}$
\\
${t_{\text{final}}}$ & Final time & ${\text{s}}$
\\
${t_{\text{step}}}$ & Time step for simulation & ${\text{s}}$
\\
${ρ_{\text{W}}}$ & Density of water & $\frac{\text{kg}}{\text{m}^{3}}$
\\
\bottomrule
\caption{Required Inputs following \hyperref[inputInitVals]{FR:Input-Initial-Values}}
\label{Table:ReqInputs}
\end{longtabu}

\section{Likely Changes}
\label{Sec:LCs}
This section lists the likely changes to be made to the software.

\begin{itemize}
\item[Temperature-Coil-Variable-Over-Day:\phantomsection\label{likeChgTCVOD}]{\hyperref[assumpTHCCoT]{A:Temp-Heating-Coil-Constant-over-Time} - The temperature of the heating coil will change over the course of the day, depending on the energy received from the sun.}
\item[Temperature-Coil-Variable-Over-Length:\phantomsection\label{likeChgTCVOL}]{\hyperref[assumpTHCCoL]{A:Temp-Heating-Coil-Constant-over-Length} - The temperature of the heating coil will actually change along its length as the water within it cools.}
\item[Discharging-Tank:\phantomsection\label{likeChgDT}]{\hyperref[assumpCTNTD]{A:Charging-Tank-No-Temp-Discharge} - The model currently only accounts for charging of the tank. That is, increasing the temperature of the water to match the temperature of the coil. A more complete model would also account for discharging of the tank.}
\item[Tank-Lose-Heat:\phantomsection\label{likeChgTLH}]{\hyperref[assumpPIT]{A:Perfect-Insulation-Tank} - Any real tank cannot be perfectly insulated and will lose heat.}
\end{itemize}

\section{Unlikely Changes}
\label{Sec:UCs}
This section lists the unlikely changes to be made to the software.

\begin{itemize}
\item[Water-Fixed-States:\phantomsection\label{unlikeChgWFS}]{\hyperref[assumpWAL]{A:Water-Always-Liquid} - It is unlikely for the change of water from liquid to a solid, or from liquid to gas to be considered.}
\item[No-Internal-Heat-Generation:\phantomsection\label{unlikeChgNIHG}]{\hyperref[assumpNIHGBW]{A:No-Internal-Heat-Generation-By-Water} - Is used for the derivations of \hyperref[FT:eBalanceOnWtr]{FT:eBalanceOnWtr}.}
\end{itemize}

\end{document}